\documentclass[a4paper,11pt]{article}
\usepackage[T1]{fontenc}
\usepackage[utf8]{inputenc}
\usepackage{lmodern}
\usepackage{listings}
\title{python-ssh help}
\author{lichun.william@gmail.com}

\begin{document}

\maketitle
\tableofcontents


\section{python-ssh.py}
\lstset{language=Python}
\begin{lstlisting}
#!/usr/bin/env python
#-*- coding = utf-8 -*-
''' the Parallel batch command running environment'''
'''
    Parallel LHCK utility
        
    Author      : Wei Lichun<lichun.william@gmail.com>
    Create Date : Thu Sep  8 21:58:04 CST 2011
    Version     : 0.9
    Recent Changes:
        support max parallel thread argument
        support login_name argument
        support regex pattern filter, invert match
'''

'''


This is free software; see the source for copying conditions. 
There is NO warranty; not even for MERCHANTABILITY or FITNESS
FOR A PARTICULAR PURPOSE.

'''
import subprocess
import optparse
import sys
import getpass
import copy
import re
import signal

import process_thread
import get_host_list


parser=optparse.OptionParser(
usage="%prog [ -p <max parallel thread number> ] -f filename [ -l login_name ] command",
version='%prog 1.2',epilog="Report any bugs to lichun.william@gmail.com", prog='python-ssh')
parser.add_option("-p","--parallel"   
  ,action="store",type="int",dest="parallel",default=10,help="max number of parallel threads ,default is 10")
parser.add_option("-f","--file"   ,action="store",type="string",dest="filename",help="the host file which stores the host list")
parser.add_option("-l","--login_name",action="store",type="string",dest="login_name",help="Specifies the user to log in as on the remote machine.  This also may be specified on a per-host basis in the configuration file")
parser.add_option("-X","--extra-arg",action="store",type="string",dest="extra_argument",help="Extra command-line argument. for example: -o ConnectTimeOut=10")
parser.add_option("-e","--regexp",metavar="PATTERN",action="store",type="string",dest="pattern",help="Use PATTERN as the pattern; useful to protect patterns beginning with -.")
parser.add_option("-v","--invert-match",action="store_false",dest="invert",default=True,help="Invert the sense of matching, to select non-matching lines.")


(options,command)=parser.parse_args()

filename=options.filename
login_name=options.login_name
pattern=options.pattern
parallel=options.parallel
extra_argument=options.extra_argument
sshpass_cmd='sshpass'

if (not filename):
    parser.error ('filename argument is required')

if (not command):
    parser.error ('command argument is required')

hostnames=get_host_list.list_host_from_file(filename)

if(login_name):
    password=getpass.getpass()

def signal_handler(signal, frame):
        print 'Ctrl+C Caught, Exiting..'
        sys.exit(1)

if __name__ == '__main__':
    tasks=[]
    if(not hostnames):
        print ('No Host found from lh util')
        sys.exit(-3)
    command.insert(0,'ssh')
    host_insert_index=1
    if (extra_argument):
        command.insert(1,extra_argument)
        host_insert_index=host_insert_index+1
    if(login_name):
        command.insert(1,login_name)
        command.insert(1,'-l')
        command.insert(0,password)
        command.insert(0,'-p')
        command.insert(0,sshpass_cmd)
        host_insert_index=host_insert_index+3
    task_group=process_thread.TaskGroup(parallel)
    for host in hostnames:
        ssh_command=copy.copy(command)
        ssh_command.insert(host_insert_index,host)
        task_group.add_task(host,ssh_command)

    signal.signal(signal.SIGINT, signal_handler)
    task_group.start()
    size=len(hostnames)
    index=0
    if(pattern):
        compiled_pattern=re.compile(pattern)
    while(index<size):
        index=index+1
        task=task_group.done_queue.get()
        if(pattern):
            if(compiled_pattern.search(task.stdout)):
                matched=True
            else:
                matched=False
            if(not(matched ^ options.invert)):
                print task.key 
        else:
                print "\033[0;36;40m",index,"of",size,"\033[0;32;40m: =============== \033[0;33;40m",task.key," \033[0;32;40m===============\033[0m"
                if(task.stdout):
                        print task.stdout,
\end{lstlisting}
\end{document}
